% Please do not change the document class
\documentclass{scrartcl}

% Please do not change these packages
\usepackage[hidelinks]{hyperref}
\usepackage[none]{hyphenat}
\usepackage{setspace}
\doublespace

% You may add additional packages here
\usepackage{amsmath}

% Please include a clear, concise, and descriptive title
\title{Improving My Game Development Practices}

% Please do not change the subtitle
\subtitle{COMP150 - CPD Report}

% Please put your student number in the author field
\author{1806868}

\begin{document}

\maketitle

\section{Introduction}

I am looking to become a computing professional who focuses on game development furthermore I would like to run my own indie studio that specializes in audio and AI/machine learning. To achieve this there is several skills that I need to continuously improve on. I would like to be able to do more in-depth code reviews to better support my team, dive deeper in to math by induction to support me with creating complex algorithms. I also need to improve my research organization skills, so I can write in more depth about the subject, I also need to gain more confidence while talking during presentations to prevent me from panicking and forgetting my lines when it’s my time to talk and I feel that I would also benefit from learning new techniques to improve my time management skills during studio hours.

\section{Code Review}

Throughout the group development project, I found that it was hard to do an in-depth code review and often found that I missed stuff that I noticed later. To achieve becoming a successful software developer, I need to be able to give clear and precise recommendation to aid my colleges. This has resulted in me missing obvious bugs that have been difficult to find further down the line. With a bit of practice, learning some new techniques I can overcome this to better guide my team in the future. 

In-order to improve this I will read ``Why Code reviews matter'' \cite{codeReviewsMatter}, ``10 ways to improve your code reviews'' \cite{waysToImprove} and read other peoples code reviews on StackExchange.com \cite{stackexchange}. After I have reviewed my team mates code, I will ask them for feedback to see if it helped them in any way. I am already able to give basic feedback, but I want to be able to give richer and more precise feedback to improve the quality of our code base and better guide my team in the future. I am aiming to gain these skills by the end of our next game project.


\section{Math By Induction}

In comp-110 I found that math by induction was extremely challenging. As an aspiring computing professional, I would benefit from mastering induction to aid me in proving that my algorithms are correct. However, I seem to get lost half way through the equations which results in me getting stressed because I don’t fully understand how it works. I know that it can be achieved in several simple steps, but I am having difficulty recalling the order of them. I know that I can overcome this challenge if I put more focus into it and go at a slower pace to make sure that I fully understand the procedures step by step.

To correct my understanding, I will complete a short math course on Udemy \cite{Udemy} which covers math induction. Once I have completed the induction lessons, I can book a meeting with a tutor to check that I can implement the skills correctly. I understand why induction is useful to prove that a math equation is correct. This will allow me to check that the math behind my algorithm is correct before I begin programming it. I intend to have achieved this by the end of the second semester. 

\section{Research Organization}

While I have been doing research for both the agile essay and the research journal, I found that my research became very disorganized. To become part of a successful computing research and development team I will require that I keep my research organized. My current system becomes a mess and I tend to forget where a certain piece of information that I’m am intending to write about is, making it difficult for me to cite my work to back up the statement that I am writing. I usually end up having to skim all the papers that I have already read to find the information to correctly cite my work.

To help me keep my work organized, I will read ``What’s the best way to organize my research?'' \cite{research} and other relevant posts online. This will enable me to write in more detail and should be reflected in feedback that I get from future essays that require academic research. I have already started implementing new methods since the agile essay that have proven successful, but I still need to find a better solution to notes organized to help me recall information at a later data. I wish to have implemented a new system by the end of my next research essay.

\section{Presentations}

During game pitches I have found it difficult to communicate my game presentation. As a keen indie developer, I will benefit from learning the art of game pitches to successfully secure funding on behalf of the studio. However, I often panic resulting in me talking to fast, stalling and forgetting my lines. I know that there is lot of material online that can help me to be better prepared in this situation. This is an issue that I can overcome if I put more practice and time into my pitches. I have previously hoped that my colleagues part of the pitch has been strong enough that my part gets slightly overlooked.

So, to help me build my pitching skills I will watch presentation on YouTube \cite{youTube} and GDC \cite{gdc}. This will be reflected in future makes and feedback that I get while at university. I am currently familiar with creating a basic pitch deck but I want to build my confidence while presenting it. This will give me a better chance of success when pitching to investors on future projects. I aim to have completed this by the end of the current academic year.


\section{Better Studio Time Management}

During studio practice week I found it hard to manage my time between code reviews, development, and supporting my team members.  As an aspiring game developer, I must develop the skills to manage my time efficiently within the working day. I almost always find myself just doing code review or supporting my team with fixing conflicts and neglecting my own development tasks. Once I fall behind, I struggle to catch up and end up doing most of my own development in my own time. I am aware that there are lots of tools, books, blogs and videos online that can help me to learn new methods to better manage my time. This is a challenge that would benefit from mastering if I done more research and found the right tool for me.

In-order for me to improve this I will watch videos on YouTube and will take notes from Randy Pausch video \cite{randy}, also I will read various blogs and post online. I will keep a time journal for working hours, at the end of the working week I will identify areas that I can improve on and devise a plan to better manage my time. I am already familiar with versus methods, but I require other techniques to help me improve it further. This will help me to achieve more throughout the working day. I will have completed this by the end of our next game project.  


\section{Conclusion}
By focusing on completing the math course on Udamy it will help to write complex math algorithms correctly and reading how to improve my code reviews will allow me to advise my team in a much more structured manner overall this will lead to our team having much cleaner code base. By improving my time management in the studio it will allow me to put more time into other assignments and increasing my confidents with pitches will aid me in presenting our next game. If I enhance my research organization, I will become more efficient with my writing and be able to write in more detail.


\bibliographystyle{IEEEtran}
\bibliography{references}

\end{document}